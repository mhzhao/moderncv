%% start of file `template.tex'.
%% Copyright 2006-2013 Xavier Danaux (xdanaux@gmail.com).
%
% This work may be distributed and/or modified under the
% conditions of the LaTeX Project Public License version 1.3c,
% available at http://www.latex-project.org/lppl/.


\documentclass[11pt,a4paper,sans]{moderncv}        % possible options include font size ('10pt', '11pt' and '12pt'), paper size ('a4paper', 'letterpaper', 'a5paper', 'legalpaper', 'executivepaper' and 'landscape') and font family ('sans' and 'roman')

\usepackage{array}
\usepackage{fontspec}
\usepackage{fontawesome}
\usepackage{pifont}
\usepackage{fancyhdr}
\usepackage{fancyvrb}
\usepackage{xcolor}
\usepackage{colortbl}
%\usepackage{minted}
%\usepackage{listings}
\usepackage{multirow}
\usepackage{physics}
\usepackage{amssymb}
\usepackage[bottom,stable]{footmisc}
\usepackage{lastpage}
\setlength{\headheight}{15.2pt}
\pagestyle{fancy}
\lhead{Q.F.T}
%\chead{{\fontspec[]{CALLIG15}\large{curriculum}} vit\ae{}}
\chead{HW5}
\rhead{October 6, 2015}
\rfoot{\thepage / \pageref{LastPage}}
\renewcommand{\headrulewidth}{3.0pt}
%\newcommand{\cmd}[1]{{\bf \color{red}#1}} % highlights command
%\newcommand\faSkype{{\FA\symbol{"F17E}}}
%\newcommand\faWeibo{{\FA\symbol{"F18A}}}
%\newcommand\faRenren{{\FA\symbol{"F18B}}}
\RecustomVerbatimCommand{\VerbatimInput}{VerbatimInput}%
{fontsize=\footnotesize,
	 %
	 frame=lines,  % top and bottom rule only
	  framesep=2em, % separation between frame and text
	   rulecolor=\color{gray},
	    %
	    label=\fbox{\color{black}Results.txt},
	     labelposition=topline,
	      %
	      commandchars=\|\(\), % escape character and argument delimiters for
	                            % commands within the verbatim
	       commentchar=*        % comment character
       }
% moderncv themes
\moderncvstyle{banking}                             % style options are 'casual' (default), 'classic', 'oldstyle' and 'banking'
\moderncvcolor{red}                               % color options 'blue' (default), 'orange', 'green', 'red', 'purple', 'grey' and 'black'
%\renewcommand{\familydefault}{\sfdefault}         % to set the default font; use '\sfdefault' for the default sans serif font, '\rmdefault' for the default roman one, or any tex font name
%\nopagenumbers{}                                  % uncomment to suppress automatic page numbering for CVs longer than one page

% character encoding
%\usepackage[utf8]{inputenc}                       % if you are not using xelatex ou lualatex, replace by the encoding you are using
%\usepackage{CJKutf8}                              % if you need to use CJK to typeset your resume in Chinese, Japanese or Korean

%\renewcommand{\theFancyVerbLine}{\sffamily\textcolor[rgb]{0.5,0.5,1.0}{\tiny\oldstylenums{\arabic{FancyVerbLine}}}}
%\usemintedstyle{emacs} % possible styles: monokai, manni, colorful, default, trac, tango, fruity, autumn, emacs, vim

%\definecolor{bg1}{rgb}{1.0,0.95,0.95}
%\definecolor{bg2}{rgb}{0.95,1.0,0.95}
%\definecolor{bg3}{rgb}{0.95,0.95,0.95}
%\definecolor{bg4}{rgb}{0.00,0.00,0.00}

\newcolumntype{L}[1]{>{\raggedright\let\newline\\\arraybackslash\hspace{0pt}}m{#1}}
\newcolumntype{C}[1]{>{\columncolor{black}\color{white}\centering\let\newline\\\arraybackslash\hspace{0pt}}m{#1}}
\newcolumntype{R}[1]{>{\raggedleft\let\newline\\\arraybackslash\hspace{0pt}}m{#1}}

% adjust the page margins
\usepackage[scale=0.75,bottom=0.8in,top=1.04in]{geometry}

% personal data
\name{Minghui}{Zhao}
\title{HW5}                               % optional, remove / comment the line if not wanted
\universityaddress{201 Physics Hall}{50011 Ames}{U.S}
\phone[mobile]{+1~(515)~598~6213}                   % optional, remove / comment the line if not wanted; the optional "type" of the phone can be "mobile" (default), "fixed" or "fax"
\email{mhzhao@iastate.edu}                               % optional, remove / comment the line if not wanted

\makeatletter
\renewcommand*{\bibliographyitemlabel}{\@biblabel{\arabic{enumiv}}}
\makeatother
\renewcommand*{\bibliographyitemlabel}{[\arabic{enumiv}]}% CONSIDER REPLACING THE ABOVE BY THIS

% bibliography with mutiple entries
%\usepackage{multibib}
%\newcites{book,misc}{{Books},{Others}}
%----------------------------------------------------------------------------------
%            content
%----------------------------------------------------------------------------------
\begin{document}
%\begin{CJK*}{UTF8}{gbsn}                          % to typeset your resume in Chinese using CJK
%-----       resume       ---------------------------------------------------------
%\begin{minipage}[t]{\textwidth}
\makecvtitle

\vspace{-1.2cm}
\section{5.a Langrangian Problem}
\large
What we have
\begin{equation*}
\begin{split}
	&\mathcal{L} = -\frac{1}{2}\partial^{\mu}\varphi\partial_{\mu}\varphi -\frac{1}{2}m^2\varphi^2 + \frac{1}{6}g\varphi^3 + \frac{1}{24}h\varphi^4
\\
	where
\\
	&  \expval{\varphi(0)}{0} = v
\\
	&  \matrixel{p}{\varphi(0)}{0} = u
\end{split}
\end{equation*}
Now we want to get a Langrangian density with a normalized $\varphi'$ to statisfy
\begin{equation*}
\begin{split}
	&  \expval{\varphi(0)'}{0} = 0
\\
	&  \matrixel{p}{\varphi(0)'}{0} = 1
\end{split}
\end{equation*}
What I do here is tranfering $\varphi$ to $\varphi'$ like this
\begin{equation*}
	\varphi' = \frac{\varphi - v}{u} \Longrightarrow \varphi = u\varphi' + v
\end{equation*}
substitute $\varphi$ using $\varphi'$ into the Lagrangian density equation, we get
\begin{equation*}
\begin{split}
	\mathcal{L} = & -\frac{1}{2}\partial^{\mu}\varphi\partial_{\mu}\varphi - \frac{1}{2}m^2{\varphi}^2 + \frac{1}{6}g{\varphi}^3 + \frac{1}{24}h{\varphi}^4
	\\
	= & -\frac{1}{2}\partial^{\mu}(u\varphi' + v)\partial_{\mu}(u\varphi' + v) - \frac{1}{2}m^2(u\varphi' + v)^2 + \frac{1}{6}g(u\varphi' + v)^3 + \frac{1}{24}h(u\varphi' + v)^4
	\\
	= & -\frac{1}{2}{u^2}\partial^{\mu}\varphi'\partial_{\mu}\varphi' - \frac{v^2}{2}
	\\
	&  - \frac{m^2u^2{\varphi'}^2}{2} - {m^2uv\varphi'} - \frac{m^2v^2}{2}
	\\
	& + \frac{gu^3{\varphi'}^3}{6} + \frac{gu^2v{\varphi'}^2}{2} + \frac{guv^2\varphi'}{2} + \frac{gv^3}{6}
	\\
	& + \frac{hu^4{\varphi'}^4}{24} + \frac{hu^3v{\varphi'}^3}{6} + \frac{hu^2v^2{\varphi'}^2}{4} + \frac{huv^3\varphi'}{6} + \frac{hv^4}{24}
	\\
	= & -\frac{1}{2}Z_{\varphi'}\partial^{\mu}\varphi'\partial_{\mu}\varphi -\frac{1}{2}m^2Z_{m}{\varphi'}^2 + \frac{1}{6}Z_{g}g{\varphi'}^3 + \frac{1}{24}Z_{h}h{\varphi'}^4 + Y\varphi' + \Omega_{0}
\end{split}
\end{equation*}
Where 
\begin{equation*}
\begin{split}
	Z_{\varphi} & = {u^2}
	\\
	Z_{m} & = {u^2} - \frac{gu^2v}{m^2} - \frac{hu^2v^2}{2m^2}
	\\
	Z_{g} & = {u^3} + \frac{hu^3v}{g}
	\\
	Z_{h} & = {u^4}
	\\
	Y & = -{m^2uv} + \frac{guv^2}{2} + \frac{huv^3}{6}
	\\
	\Omega_{0} & = -\frac{v^2}{2} - \frac{m^2v^2}{2} + \frac{gv^3}{6} + \frac{hv^4}{24}
\end{split}
\end{equation*}
\clearpage
\section{5.b Method Of Steepest Descents\textcolor{black}{\footnotemark[1]}}
\enlargethispage{\footskip}
\footnotetext[1]{Arfken, G. "Method of Steepest Descents." \S 12.7 in \textit{Mathematical Methods for Physicists, 7th ed.} Elsevier pp. 585-589, 2013}
Consider the asymptotic behavior (for large $t$, assumed real) of a function $f(t)$, where
\vspace{0.2cm}
\begin{itemize}
	\item $f(t)$ is represented by an integral of the generic form
		\begin{equation*}
			f(t) = \int\limits_{C} F(z,t) dz,
		\end{equation*}
		with $F(z,t)$ analytic in $z$, but also parametrically dependent on $t$;
	\item The integration path $C$ is, or can be deformed to be, such that for large $t$ the dominant contribution to the integral arises from a small range of $z$ in the neighborhood of the point $z_{0}$ where $|F(z_{0},t)|$ is a maximum \textbf{on the path;}
	\item The integration path will pass through $z_{0}$ in the orientation that causes the most rapid decrease in $|F|$ on departure from $z_{0}$ in either direction along the path (hence the name \textbf{steepest descents}); and 
	\item In the limit of large $t$ the contribution to the integral from the neighborhood of $z_{0}$ asymptotically approaches the exact value of $f(t)$.
\end{itemize}
\vspace{0.2cm}
We call $z_{0}$ as saddle point. For analytic functions, they do not have maxima and minima of their moduli in the range of their analyticity, they have only saddle points.
Once we identify $z_{0}$ and the directions of steepest desent in $|F(z,t)|$, we complete the specification of the method of steepest descents, also called the \textbf{saddle point method} of asymptotic approximation, by assuming that the significant contributions to the integral are from a small range of $0 \le r \le a$ in each of the two directions along the path.

\vspace{0.2cm}
For example, take the form 
\begin{equation*}
	F(z,t) = e^{w(z,t)} = e^{u(z,t) + iv(z,t)}
\end{equation*}
if the contour $C$ can be deformed in such a way that it passes through the saddle point in the direction of the steepest descent, 
the large value of parameter $t$ ensures that the essential contribution to the integral comes from a small vicinity of the saddle point since the exponent quickly decreaseds and the detailed shape of the curve of the steepest decent far from the saddle point is not important. 
In the saddle point the derivative $w' = 0$. The direction of the steepest descent is determined by the requirement that the difference $u(z,t) - u(z_{0},t)$ is negative. By Taylor expanding $w(z,t)$ about $z_{0}$ and introduing polar forms, we can get 
\begin{equation*}
	f(t) \approx 2e^{w_{0} + i\theta}\int\limits^{a}_{0} e^{-|w_{0}''|r^2/2} dr
\end{equation*}
Now make the key assumption of the method, namely that $|w_{0}''|$, 
the measure of the rate of decrease in $|F|$ as we leave $z_{0}$, 
is large enough that the bulk of the value of the integral has already been attained for small $a$, 
and that the exponential decreasein the value of the intgrand enables us to replace $a$ by infinity without making significant error. 
Take $a = \infty$, get the value $\sqrt{\pi/2|w_{0}''|}$ of the integral. Then we get
\begin{equation*}
	f(t) \approx F(z_{0},t)e^{i\theta}\sqrt{\frac{2\pi}{|w''(z_{0},t)|}}
\end{equation*}
\vspace{0.2cm}
Sometimes it is sufficient to apply the method of steepest descents only to the rapidly varying of an integral. 
\begin{equation*}
	f(t) = \int\limits_{C} g(z,t)F(z,t)dz \approx g(z_{0},t)\int\limits_{C} F(z,t)dz
\end{equation*}
\end{document}


%% end of file `template.tex'.
